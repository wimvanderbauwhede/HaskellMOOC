%-----------------------------------------------------------------------
% Functional Programming 4
% John O'Donnell, Wim Vanderbauwhede
% University of Glasgow
%-----------------------------------------------------------------------

\documentclass{beamer}
\usepackage{jtodlecseriesFP4}
%include polycode.fmt
%format alpha = "\alpha"
%format -- > = "\leadsto "

% Identify this presentation
\SetPresentationTitle
  {Computations and Polymorphism}
  {Computations and Polymorphism}
\SetPresentationNumber
  {4}
\SetPresentationDate
  {Week 2-2}
  {Week 2-2}

%-----------------------------------------------------------------------
% Beginning

\begin{document}

\begin{frame}[fragile]
  \PresentationTitleSlide
\end{frame}

\begin{frame}[fragile]
  \frametitle{Topics}
  \tableofcontents
\end{frame}

%-----------------------------------------------------------------------
%-----------------------------------------------------------------------
\begin{frame}[fragile]
\begin{center}
\includegraphics[scale=0.075]
	{figures/jpg/pic03.jpg}
\end{center}
\end{frame}
%-----------------------------------------------------------------------
\section{Computations}
\subsection{Running Starman}
\begin{frame}[fragile]
\frametitle{Running the Starman program}

\begin{enumerate}
\item Launch haskell: {\tt hci}
\item Load the program: {\tt :load Starman}
\item Run a game by entering \emph{starman word n}, where $word ::
  String$ is the word to be guessed, and $n :: Int$ is the number
  of stars you start with.
\item Example: \emph{starman "astonishing" 5}
\item When the game finishes, you'll be back at the ghci prompt.
\end{enumerate}
\end{frame}


%-----------------------------------------------------------------------
\begin{frame}[fragile]
\frametitle{Winning}

{\footnotesize
\begin{verbatim}
$  starman elephant 4
________  ****
  Enter your guess: e
e_e_____  ****
  Enter your guess: q
e_e_____  ***
  Enter your guess: r
e_e_____  **
  Enter your guess: p
e_ep____  **
  Enter your guess: t
e_ep___t  **
  Enter your guess: a
e_ep_a_t  **
  Enter your guess: n
e_ep_ant  **
  Enter your guess: l
elep_ant  **
  Enter your guess: h
You win!
\end{verbatim}
}
\end{frame}

%-----------------------------------------------------------------------
\begin{frame}[fragile]
\frametitle{Losing}

{\footnotesize
\begin{verbatim}
$ starman easy 3
____  ***
  Enter your guess: e
e___  ***
  Enter your guess: a
ea__  ***
  Enter your guess: c
ea__  **
  Enter your guess: t
ea__  *
  Enter your guess: h
You lose
\end{verbatim}
}
\end{frame}

%-----------------------------------------------------------------------
\begin{frame}[fragile]
\frametitle{Improvements}

\begin{itemize}
\item There are some small improvements.  If you win, it would be
  nice to display the complete word, but the program just says you
  won.
\item Major improvement:
  \begin{itemize}
  \item To run the game, you have to tell it the secret word.  So
    it isn't very secret.
  \item We should really generate the words automatically in the
    program: for example, we could generate a random number $i$ and
    read in the $i$th word from a dictionary.
  \end{itemize}
\end{itemize}

\end{frame}



%-----------------------------------------------------------------------
\subsection{IO operations}

\begin{frame}[fragile]
\frametitle{Computations}

\begin{itemize}
\item Most programming languages have types for primitive data:
  $Int$, $Bool$, $Double$, $Char$.
\item Most languages also have at least a few types for structured
  data, such as arrays.  (Haskell has this.)
\item Some languages go farther, and have types for more advanced
  data structures: tuples, lists, records, enumerations.  (Haskell
  has all these.)
\item A few languages go still farther, and have user-definable
  types for general structures like trees and graphs.  (Haskell has
  this.)
\item Haskell goes still farther: it has the concept of a
  \emph{computation}, and \emph{types to represent computations}.
\item Computations are ``first class'': they can be used as
  arguments to functions, or results of functions, they can be
  stored in data structures, etc.
\end{itemize}

\end{frame}

%-----------------------------------------------------------------------
\begin{frame}[fragile]
\frametitle{Computation types}

\begin{itemize}
\item A computation type has a form like $IO\; Int$.
  \begin{itemize}
  \item The first part, $IO$, is a \emph{monad}.  We'll look at
    monads later, but for now, think of a monad as the method by
    which the computations in a $do$ expression are combined into a
    larger computation.
  \item There are many monads in Haskell, and you can define your
    own.  For the time being, we will be working with just one
    monad --- $IO$ --- so all our computations will have types that
    start with $IO$.
  \item The second part ($Int$ in the example above) is the type of
    \emph{the result computed by the computation}.
  \end{itemize}
\item So $IO\; Int$ means ``the type of a computation that does some
  work and then eventually produces a value of type $Int$''.
\end{itemize}

\end{frame}

%-----------------------------------------------------------------------
\begin{frame}[fragile]
\frametitle{The type of $putStrLn$}

Here is the type of $putStrLn$.  It isn't a statement!  It's a
function that takes an argument and returns a result.

\begin{verbatim}
putStrLn :: String -> IO ()
\end{verbatim}

\begin{itemize}
\item $putStrLn$ takes a $String$ and it returns a computation.
  That computation has type $IO ()$.
\item The computation must run in the $IO$ monad.  This means you
  can combine $putStrLn$ with other $IO$ operations, but not with
  some completely unrelated computations that we'll encounter
  later.
\item When the computation executes it will perform the output.  It
  doesn't have a useful result to produce, so by convention it just
  produces the empty tuple $()$.
\item The application $putStrLn$\emph{ "hello"} has type $IO ()$ --- when
  you apply the function $putStrLn$ to a string, the result is the
  computation.
\end{itemize}

\end{frame}

%-----------------------------------------------------------------------
\begin{frame}[fragile]
\frametitle{Some text output operations}

\begin{itemize}
\item $putStr :: String -> IO ()$
\item $putStrLn :: String -> IO ()$
\item $putChar :: Char -> IO ()$
\end{itemize}

\end{frame}

%-----------------------------------------------------------------------
\begin{frame}[fragile]
\frametitle{$getChar$}

Here is the type of $getChar$:

\begin{verbatim}
getChar :: IO Char
\end{verbatim}

\begin{itemize}
\item $getChar$ is not a function: it doesn't need an arguments to
  tell it how to read a character.
\item It is simply a constant, like $42$ or $'a'$ or any other
  constant.
\item The type says that
  \begin{itemize}
  \item $getChar$ executes in the $IO$ monad --- that means that
    you can combine it in a $do$ expression with other $IO$
    operations, but not with operations that are completely
    unrelated.
  \item When the computation is executed, it yields a result with
    type $Char$.
  \end{itemize}
\end{itemize}
\end{frame}

%-----------------------------------------------------------------------
\begin{frame}[fragile]
\frametitle{Receiving the result of a computation}

\begin{itemize}
\item When the computation $getChar$ is executed, we need to grab
  the result so we can use it!
\item This is done by binding a name to the result of the computation.
\item {\tt c <- getChar} means ``run the computation $getChar$ and then
  define $c$ to be the value produced by the computation''.
\item Since $getChar :: IO Char$, we will have $c :: Char$
\item In general, if you write {\tt x <- someComputation}, where
  $someComputation :: IO a$ for any type $a$, then $x$ will have
  the type $a$.
\end{itemize}

\end{frame}

%-----------------------------------------------------------------------
\begin{frame}[fragile]
\frametitle{More text input operations}

\begin{itemize}
\item $getChar :: IO Char$
\item $getLine :: IO String$
\end{itemize}

\end{frame}


%-----------------------------------------------------------------------
\subsection{Using let in a monad}
\begin{frame}
\frametitle{Using \emph{let} in a monad}

\begin{itemize}
\item What if you want to do something ordinary in a $do$
  expression, like just say {\tt x = 2+2}?
\item You can't put an equation like this in a $do$ expression
  because it doesn't have the right monadic type.
\item Instead, you write {\tt let x = 2+2}.  This is now a computation,
  and the types are ok.
\item A typical $do$ expression combines computations (with and
  without {\tt <-} and {\tt let} expressions.
\end{itemize}

\end{frame}


%-----------------------------------------------------------------------
\subsection{show and read}

\begin{frame}[fragile]
\frametitle{The \emph{show} function}

\begin{itemize}
\item When you have done some calculation (like $2+2$) you may want
  to print it!
\item But $putStrLn$ requires a $String$ to print, and $2+2 :: Int$.
\item Solution: you can apply the function $show$ to an $Int$, and
  it converts it to a string.
\end{itemize}

\begin{verbatim}
show (2+2) -- > "4"
\end{verbatim}

\emph{Note that the result is a string, not a number.}

\end{frame}

%-----------------------------------------------------------------------
\begin{frame}[fragile]
\frametitle{The $read$ function}

\begin{itemize}
\item When you have read some input with $getLine$, it is a
  $String$.
\item You can convert it to an $Int$ (or some other type) using
  $read$.
\item But if you have read a string like "23", how can $read$ know
  whether it should give you an $Int$, or a $Double$, or one of the
  many other types it could be?
\item \emph{You need to tell it} by specifying the type of the
  result $read$ should produce.
\end{itemize}

\begin{verbatim}
read "23" :: Int     -- > 23
read "23" :: Double  -- > 23.0
\end{verbatim}

If you just write {\tt read "23"} Haskell will give an error message,
referring to ``Ambiguous type variable$\ldots$''.

\end{frame}

%-----------------------------------------------------------------------
\begin{frame}[fragile]
\frametitle{Example: program \tt{Lec04Prog1.hs}}

\begin{verbatim}
main =
  do  putStrLn "Enter a number"
      xs <- getLine
      let n = 3 * (read xs :: Int)
      putStrLn (show n)
\end{verbatim}

\begin{verbatim}
*Main> main
Enter a number
21
63
\end{verbatim}

\end{frame}

%-----------------------------------------------------------------------
\section{Polymorphic types}

\begin{frame}[fragile]
\frametitle{Polymorphic types}

\begin{itemize}
\item The word \emph{polymorphic} means ``has many forms''.
\item In Haskell, a polymorphic type is a set of basic types.
\item Polymorphic functions raise the level of abstraction, and
  they make programs more concise.
\end{itemize}

\end{frame}


%-----------------------------------------------------------------------
\begin{frame}[fragile]
\frametitle{$length$ of a list}

To motivate the need for polymorphism, we look at the $length$
function.

\begin{itemize}
\item $length$ takes a list, and returns the number of elements.
\item {\tt length [2,5,6,7] -- > 4}
\item {\tt length [] -- > 0}
\end{itemize}

There are many mathematical theorems about  functions, and
here is one of them.

\begin{theorem}
  For all $xs, \,ys\, ::\, [a]$, $length\, (xs++ys)\, =\, length\,  xs + length\,ys$.
\end{theorem}

Later, we will prove this theorem.

\end{frame}

%-----------------------------------------------------------------------
\begin{frame}[fragile]
\frametitle{What is the type of \emph{length}?}

Consider some examples:

\begin{itemize}
\item \texttt{length [4,8,2] -- > 3}, so apparently {\tt length :: [Int] -> Int}.
\item But \texttt{length [3.1, 6,7] -- > 2}, so now it seems that {\tt length ::
  [Double] -> Int}.
\item Also \texttt{length ["a","bb","ccc"'] -- > 3}, which would imply that
  {\tt length :: [String] -> Int}.
\item This isn't all: \emph{for any type at all, if you have a list
    of elements of that type, $length$ will take that list and
    return an $Int$}.
\end{itemize}

{\redtext Conclusion: $length$ has many types: it is
  \emph{polymorphic.}}

\end{frame}

%-----------------------------------------------------------------------
\begin{frame}[fragile]
\frametitle{Type variables}

\begin{itemize}
\item A concrete type is just one specific datatype: $Int$,
  $Double$, $Bool$, etc.
\item A type variable can stand for any arbitrary type: $a$, $b$,
  etc.
\item A polymorphic function's type is expressed using a type
  variable.
\end{itemize}

\begin{verbatim}
length :: [a] -> Int
\end{verbatim}

This means: for any type $a$, the function $length$ takes a list
whose elements have type $a$ and returns an $Int$.

\end{frame}

%-----------------------------------------------------------------------
\begin{frame}[fragile]
\frametitle{The type of $++$}

\begin{itemize}
\item The $++$ operator is just a function: it takes two lists and
  returns a big list containing all their elements.
\item These lists could have \emph{any type at all}
  \begin{itemize}
  \item \texttt{[2,3] ++ [6,7,8] -- > [2,3,6,7,8]}.  Here all the lists have
    type $[Int]$.
  \item \texttt{["head",1] ++ ["strong",2] -- > ["head",1,"strong",2]}.  Here all the lists
    have "mixed" types. This would be a type error in Haskell!
  \end{itemize}
\end{itemize}

\begin{verbatim}
(++) :: [a] -> [a] -> [a]
\end{verbatim}

\end{frame}

%-----------------------------------------------------------------------
\begin{frame}[fragile]
\frametitle{Saying that two types must be the same}

\begin{itemize}
\item To say that two types are arbitrary, give them
  \emph{different type variable names}
  \begin{itemize}
  \item $f :: [a] -> [b] -> Int$ says that $f$ takes two lists, and
    each list could have any type at all.
  \end{itemize}
\item To say that two types are arbitrary except that they have to
  be the same, give them \emph{the same type variable name}.
  \begin{itemize}
  \item $g :: [a] -> [a] -> Int$ says that $g$ takes two lists, and
    they can have any element type except that the two lists must
    be the \emph{same} type.
  \end{itemize}
  \item The type variable $a$ can refer to a polymorphic type that contains all possible types. This is the case in e.g. LiveScript arrays: the array can contain values of any type, so the type variable must refer to a type that contains all possible types. You will find this notation used in the LiveScript Prelude documentation.
\end{itemize}

\end{frame}

%-----------------------------------------------------------------------
\begin{frame}[fragile]
\frametitle{Type of \emph{zip}}

\begin{itemize}
\item The $zip$ function takes two lists, and they can have any
  type at all.  Therefore their types are described with two
  different type variables.
\item It returns a list of pairs, with each element of the pair
  coming from one of the lists.  This constrains the output type,
  so we say the result type is $[(a,b)]$.
\end{itemize}

\begin{verbatim}
zip :: [a] -> [b] -> [ (a, b) ]
\end{verbatim}

\end{frame}

%-----------------------------------------------------------------------
\section{Using ghci}

\begin{frame}[fragile]
\frametitle{Haskell script files}

Create a file named {\tt Basic1.hs} with the following contents:

%\vspace{1em}
%
%\fbox{
%\begin{minipage}{8cm}
\begin{verbatim}
g :: Int -> Int
g x = x+2
\end{verbatim}
%\end{minipage}
%}
%\vspace{1em}

Use the {\tt load} command to read the script file:

{\footnotesize
\begin{verbatim}
Prelude> :load Basic1
[1 of 1] Compiling Main             ( Basic1.hs, interpreted )
Ok, modules loaded: Main.
\end{verbatim}
}

\begin{verbatim}
*Main> g 15
17
\end{verbatim}

\end{frame}

%-----------------------------------------------------------------------
\begin{frame}[fragile]
\frametitle{Asking ghci about types}

\begin{itemize}
\item ghci has a command {\tt :t} --- put an expression after{\tt :t}  and
  ghci will tell you its type.
\end{itemize}

\begin{verbatim}
Prelude> :t (2<3)
(2<3) :: Bool
\end{verbatim}

Useful for finding types of functions.

\begin{verbatim}
Prelude> :t zip
zip :: [a] -> [b] -> [(a, b)]
\end{verbatim}

Sometimes it may surprise you!

\begin{verbatim}
Prelude> :t 3
3 :: (Num t) => t
\end{verbatim}

{\redtext There are a lot of Haskell features we haven't seen yet
  (but we will soon).  But for now, this means you'll see types and
  error messages that aren't meaningful because they refer to some
  of these features!}

\end{frame}


\end{document}
