\subsection{Partial Application and
Currying}\label{partial-application-and-currying}

\subsubsection{Currying}\label{currying}

Consider a type signature of a function with three arguments:

\begin{Shaded}
\begin{Highlighting}[]
\OtherTok{    f ::} \DataTypeTok{X} \OtherTok{->} \DataTypeTok{Y} \OtherTok{->} \DataTypeTok{Z} \OtherTok{->} \DataTypeTok{A}
\end{Highlighting}
\end{Shaded}

The arrow ``-\textgreater{}'' is right-associative, so this is the same
as:

\begin{Shaded}
\begin{Highlighting}[]
\OtherTok{    f ::} \DataTypeTok{X} \OtherTok{->} \NormalTok{(}\DataTypeTok{Y} \OtherTok{->} \NormalTok{(}\DataTypeTok{Z} \OtherTok{->} \DataTypeTok{A}\NormalTok{))}
\end{Highlighting}
\end{Shaded}

What this means is that we can consider \texttt{f} as a function with a
single argument of type \texttt{X} which returns a function of type
\texttt{Y-\textgreater{}Z-\textgreater{}A}.

The technique of rewriting a function of multiple arguments into a
sequence of functions with a single argument is called \emph{currying}.
We can illustrate this best using a lambda function:

\begin{Shaded}
\begin{Highlighting}[]
    \NormalTok{\textbackslash{}x y z }\OtherTok{->} \FunctionTok{...}
    \NormalTok{\textbackslash{}x }\OtherTok{->} \NormalTok{(\textbackslash{}y z }\OtherTok{->} \FunctionTok{...}\NormalTok{)}
    \NormalTok{\textbackslash{}x }\OtherTok{->} \NormalTok{(\textbackslash{}y }\OtherTok{->} \NormalTok{(\textbackslash{}z }\OtherTok{->} \FunctionTok{...} \NormalTok{))  }
\end{Highlighting}
\end{Shaded}

The name ``currying'', is a reference to logician Haskell Curry. The
concept was actually proposed originally by another logician, Moses
Schönfinkel, but his name was not so catchy.

\subsubsection{Partial application}\label{partial-application}

Partial application means that we don't need to provide all arguments to
a function. For example, given

\begin{Shaded}
\begin{Highlighting}[]
    \NormalTok{sq x y }\FunctionTok{=} \NormalTok{x}\FunctionTok{*}\NormalTok{x}\FunctionTok{+}\NormalTok{y}\FunctionTok{*}\NormalTok{y}
\end{Highlighting}
\end{Shaded}

We note that function application associates to the left, so the
following equivalence holds

\begin{Shaded}
\begin{Highlighting}[]
    \NormalTok{sq x y }\FunctionTok{=} \NormalTok{(sq x) y}
\end{Highlighting}
\end{Shaded}

We can therefore create a specialised function by partial application of
x:

\begin{Shaded}
\begin{Highlighting}[]
    \NormalTok{sq4 }\FunctionTok{=} \NormalTok{sq }\DecValTok{4} \CommentTok{-- = \textbackslash{}y -> 16+y*y}
    \NormalTok{sq4 }\DecValTok{3} \CommentTok{-- = (sq 4) 3 = sq 4 3 = 25}
\end{Highlighting}
\end{Shaded}

This is why you can write things like:

\begin{Shaded}
\begin{Highlighting}[]
    \NormalTok{doubles }\FunctionTok{=} \NormalTok{map (}\FunctionTok{*}\DecValTok{2}\NormalTok{) [}\DecValTok{1} \FunctionTok{..} \NormalTok{]}
\end{Highlighting}
\end{Shaded}

