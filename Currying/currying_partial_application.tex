\documentclass[11pt]{amsart}
\usepackage{geometry}                % See geometry.pdf to learn the layout options. There are lots.
\geometry{letterpaper}                   % ... or a4paper or a5paper or ... 
%\geometry{landscape}                % Activate for for rotated page geometry
%\usepackage[parfill]{parskip}    % Activate to begin paragraphs with an empty line rather than an indent
%\usepackage{hyperref}
\usepackage{graphicx}
\usepackage{amssymb}
\usepackage{epstopdf}
\DeclareGraphicsRule{.tif}{png}{.png}{`convert #1 `dirname #1`/`basename #1 .tif`.png}
% From https://gist.github.com/cormacrelf/3760427
\usepackage{color}
\usepackage{fancyvrb}
\DefineShortVerb[commandchars=\\\{\}]{\|}
\DefineVerbatimEnvironment{Highlighting}{Verbatim}{commandchars=\\\{\}}
% Add ',fontsize=\small' for more characters per line
\newenvironment{Shaded}{}{}
\newcommand{\KeywordTok}[1]{\textcolor[rgb]{0.00,0.44,0.13}{\textbf{{#1}}}}
\newcommand{\DataTypeTok}[1]{\textcolor[rgb]{0.56,0.13,0.00}{{#1}}}
\newcommand{\DecValTok}[1]{\textcolor[rgb]{0.25,0.63,0.44}{{#1}}}
\newcommand{\BaseNTok}[1]{\textcolor[rgb]{0.25,0.63,0.44}{{#1}}}
\newcommand{\FloatTok}[1]{\textcolor[rgb]{0.25,0.63,0.44}{{#1}}}
\newcommand{\CharTok}[1]{\textcolor[rgb]{0.25,0.44,0.63}{{#1}}}
\newcommand{\StringTok}[1]{\textcolor[rgb]{0.25,0.44,0.63}{{#1}}}
\newcommand{\CommentTok}[1]{\textcolor[rgb]{0.38,0.63,0.69}{\textit{{#1}}}}
\newcommand{\OtherTok}[1]{\textcolor[rgb]{0.00,0.44,0.13}{{#1}}}
\newcommand{\AlertTok}[1]{\textcolor[rgb]{1.00,0.00,0.00}{\textbf{{#1}}}}
\newcommand{\FunctionTok}[1]{\textcolor[rgb]{0.02,0.16,0.49}{{#1}}}
\newcommand{\RegionMarkerTok}[1]{{#1}}
\newcommand{\ErrorTok}[1]{\textcolor[rgb]{1.00,0.00,0.00}{\textbf{{#1}}}}
\newcommand{\NormalTok}[1]{{#1}}
\ifxetex
  \usepackage[setpagesize=false, % page size defined by xetex
              unicode=false, % unicode breaks when used with xetex
              xetex,
              colorlinks=true,
              linkcolor=blue]{hyperref}
\else
  \usepackage[unicode=true,
              colorlinks=true,
              linkcolor=blue]{hyperref}
\fi
\hypersetup{breaklinks=true, pdfborder={0 0 0}}
\setlength{\parindent}{0pt}
\setlength{\parskip}{6pt plus 2pt minus 1pt}
\setlength{\emergencystretch}{3em}  % prevent overfull lines
\setcounter{secnumdepth}{0}

%\EndDefineVerbatimEnvironment{Highlighting}

\title{Curry is on the menu}
\author{Wim Vanderbauwhede}
%\date{}                                           % Activate to display a given date or no date

\begin{document}
\maketitle
\subsection{Partial Application and
Currying}\label{partial-application-and-currying}

\subsubsection{Currying}\label{currying}

Consider a type signature of a function with three arguments:

\begin{Shaded}
\begin{Highlighting}[]
\OtherTok{    f ::} \DataTypeTok{X} \OtherTok{->} \DataTypeTok{Y} \OtherTok{->} \DataTypeTok{Z} \OtherTok{->} \DataTypeTok{A}
\end{Highlighting}
\end{Shaded}

The arrow ``-\textgreater{}'' is right-associative, so this is the same
as:

\begin{Shaded}
\begin{Highlighting}[]
\OtherTok{    f ::} \DataTypeTok{X} \OtherTok{->} \NormalTok{(}\DataTypeTok{Y} \OtherTok{->} \NormalTok{(}\DataTypeTok{Z} \OtherTok{->} \DataTypeTok{A}\NormalTok{))}
\end{Highlighting}
\end{Shaded}

What this means is that we can consider \texttt{f} as a function with a
single argument of type \texttt{X} which returns a function of type
\texttt{Y-\textgreater{}Z-\textgreater{}A}.

The technique of rewriting a function of multiple arguments into a
sequence of functions with a single argument is called \emph{currying}.
We can illustrate this best using a lambda function:

\begin{Shaded}
\begin{Highlighting}[]
    \NormalTok{\textbackslash{}x y z }\OtherTok{->} \FunctionTok{...}
    \NormalTok{\textbackslash{}x }\OtherTok{->} \NormalTok{(\textbackslash{}y z }\OtherTok{->} \FunctionTok{...}\NormalTok{)}
    \NormalTok{\textbackslash{}x }\OtherTok{->} \NormalTok{(\textbackslash{}y }\OtherTok{->} \NormalTok{(\textbackslash{}z }\OtherTok{->} \FunctionTok{...} \NormalTok{))  }
\end{Highlighting}
\end{Shaded}

The name ``currying'', is a reference to logician Haskell Curry. The
concept was actually proposed originally by another logician, Moses
Schönfinkel, but his name was not so catchy.

\subsubsection{Partial application}\label{partial-application}

Partial application means that we don't need to provide all arguments to
a function. For example, given

\begin{Shaded}
\begin{Highlighting}[]
    \NormalTok{sq x y }\FunctionTok{=} \NormalTok{x}\FunctionTok{*}\NormalTok{x}\FunctionTok{+}\NormalTok{y}\FunctionTok{*}\NormalTok{y}
\end{Highlighting}
\end{Shaded}

We note that function application associates to the left, so the
following equivalence holds

\begin{Shaded}
\begin{Highlighting}[]
    \NormalTok{sq x y }\FunctionTok{=} \NormalTok{(sq x) y}
\end{Highlighting}
\end{Shaded}

We can therefore create a specialised function by partial application of
x:

\begin{Shaded}
\begin{Highlighting}[]
    \NormalTok{sq4 }\FunctionTok{=} \NormalTok{sq }\DecValTok{4} \CommentTok{-- = \textbackslash{}y -> 16+y*y}
    \NormalTok{sq4 }\DecValTok{3} \CommentTok{-- = (sq 4) 3 = sq 4 3 = 25}
\end{Highlighting}
\end{Shaded}

This is why you can write things like:

\begin{Shaded}
\begin{Highlighting}[]
    \NormalTok{doubles }\FunctionTok{=} \NormalTok{map (}\FunctionTok{*}\DecValTok{2}\NormalTok{) [}\DecValTok{1} \FunctionTok{..} \NormalTok{]}
\end{Highlighting}
\end{Shaded}

\end{document}
