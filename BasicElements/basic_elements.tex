\documentclass[11pt]{amsart}
\usepackage{geometry}                % See geometry.pdf to learn the layout options. There are lots.
\geometry{letterpaper}                   % ... or a4paper or a5paper or ... 
%\geometry{landscape}                % Activate for for rotated page geometry
%\usepackage[parfill]{parskip}    % Activate to begin paragraphs with an empty line rather than an indent
%\usepackage{hyperref}
\usepackage{graphicx}
\usepackage{amssymb}
\usepackage{epstopdf}
\DeclareGraphicsRule{.tif}{png}{.png}{`convert #1 `dirname #1`/`basename #1 .tif`.png}
% From https://gist.github.com/cormacrelf/3760427
\usepackage{color}
\usepackage{fancyvrb}
\DefineShortVerb[commandchars=\\\{\}]{\|}
\DefineVerbatimEnvironment{Highlighting}{Verbatim}{commandchars=\\\{\}}
% Add ',fontsize=\small' for more characters per line
\newenvironment{Shaded}{}{}
\newcommand{\KeywordTok}[1]{\textcolor[rgb]{0.00,0.44,0.13}{\textbf{{#1}}}}
\newcommand{\DataTypeTok}[1]{\textcolor[rgb]{0.56,0.13,0.00}{{#1}}}
\newcommand{\DecValTok}[1]{\textcolor[rgb]{0.25,0.63,0.44}{{#1}}}
\newcommand{\BaseNTok}[1]{\textcolor[rgb]{0.25,0.63,0.44}{{#1}}}
\newcommand{\FloatTok}[1]{\textcolor[rgb]{0.25,0.63,0.44}{{#1}}}
\newcommand{\CharTok}[1]{\textcolor[rgb]{0.25,0.44,0.63}{{#1}}}
\newcommand{\StringTok}[1]{\textcolor[rgb]{0.25,0.44,0.63}{{#1}}}
\newcommand{\CommentTok}[1]{\textcolor[rgb]{0.38,0.63,0.69}{\textit{{#1}}}}
\newcommand{\OtherTok}[1]{\textcolor[rgb]{0.00,0.44,0.13}{{#1}}}
\newcommand{\AlertTok}[1]{\textcolor[rgb]{1.00,0.00,0.00}{\textbf{{#1}}}}
\newcommand{\FunctionTok}[1]{\textcolor[rgb]{0.02,0.16,0.49}{{#1}}}
\newcommand{\RegionMarkerTok}[1]{{#1}}
\newcommand{\ErrorTok}[1]{\textcolor[rgb]{1.00,0.00,0.00}{\textbf{{#1}}}}
\newcommand{\NormalTok}[1]{{#1}}
\ifxetex
  \usepackage[setpagesize=false, % page size defined by xetex
              unicode=false, % unicode breaks when used with xetex
              xetex,
              colorlinks=true,
              linkcolor=blue]{hyperref}
\else
  \usepackage[unicode=true,
              colorlinks=true,
              linkcolor=blue]{hyperref}
\fi
\hypersetup{breaklinks=true, pdfborder={0 0 0}}
\setlength{\parindent}{0pt}
\setlength{\parskip}{6pt plus 2pt minus 1pt}
\setlength{\emergencystretch}{3em}  % prevent overfull lines
\setcounter{secnumdepth}{0}

%\EndDefineVerbatimEnvironment{Highlighting}




\title{Basic elements of Haskell by example }
\author{Wim Vanderbauwhede}
%\date{}                                           % Activate to display a given date or no date

\begin{document}
\maketitle
%\section{}
%\subsection{}

We introduce some basic elements of Haskell through comparison with
other languages

\subsection{~Expressions}\label{expressions}

In almost all programming languages you can create \emph{expressions}
such as:

\begin{Shaded}
\begin{Highlighting}[]
    \NormalTok{(b}\FunctionTok{*}\NormalTok{b}\FunctionTok{-}\DecValTok{4}\FunctionTok{*}\NormalTok{a}\FunctionTok{*}\NormalTok{c)}\FunctionTok{/}\DecValTok{2}\FunctionTok{*}\NormalTok{a}
\end{Highlighting}
\end{Shaded}

and you can assign these expressions to variables:

\begin{Shaded}
\begin{Highlighting}[]
    \NormalTok{v }\FunctionTok{=} \NormalTok{(b}\FunctionTok{*}\NormalTok{b}\FunctionTok{-}\DecValTok{4}\FunctionTok{*}\NormalTok{a}\FunctionTok{*}\NormalTok{c)}\FunctionTok{/}\DecValTok{2}\FunctionTok{*}\NormalTok{a}
\end{Highlighting}
\end{Shaded}

In \texttt{Haskell}, you can do this as well, and what's more:
expressions are really all there is, there are no statements.

\subsection{Functions}\label{functions}

In \texttt{Python}, you can define a function such as

\begin{Shaded}
\begin{Highlighting}[]
    \KeywordTok{def} \NormalTok{hello(name):}
        \KeywordTok{return} \StringTok{"Hello, "}\NormalTok{+name}
\end{Highlighting}
\end{Shaded}

In \texttt{Haskell} you can write this simply as:

\begin{Shaded}
\begin{Highlighting}[]
    \NormalTok{hello name }\FunctionTok{=} \StringTok{"Hello, "}\FunctionTok{++}\NormalTok{name}
\end{Highlighting}
\end{Shaded}

\subsection{Types}\label{types}

\texttt{C} has \emph{types}, for example:

\begin{Shaded}
\begin{Highlighting}[]
    \DataTypeTok{int} \NormalTok{f (}\DataTypeTok{int} \NormalTok{x, }\DataTypeTok{int} \NormalTok{y) \{}
        \KeywordTok{return} \NormalTok{x*y+x+y;}
    \NormalTok{\}}
\end{Highlighting}
\end{Shaded}

\texttt{Haskell} has much more powerful types than \texttt{C}, and we
will talk a lot about types:

\begin{Shaded}
\begin{Highlighting}[]
\OtherTok{    f ::} \DataTypeTok{Int} \OtherTok{->} \DataTypeTok{Int} \OtherTok{->} \DataTypeTok{Int}
    \NormalTok{f x y }\FunctionTok{=}  \NormalTok{x}\FunctionTok{*}\NormalTok{y}\FunctionTok{+}\NormalTok{x}\FunctionTok{+}\NormalTok{y}
\end{Highlighting}
\end{Shaded}

\subsection{Lists}\label{lists}

In many languages, e.g.~Python, JavaScript, Ruby, \ldots{} you can
create \emph{lists} such as:

\begin{Shaded}
\begin{Highlighting}[]
    \NormalTok{lst }\FunctionTok{=} \NormalTok{[ }\StringTok{"A"}\NormalTok{, }\StringTok{"list"}\NormalTok{, }\StringTok{"of"}\NormalTok{, }\StringTok{"strings"}\NormalTok{]}
\end{Highlighting}
\end{Shaded}

\texttt{Haskell} also uses this syntax for lists.

To join lists, in \texttt{Python} you could write

\begin{Shaded}
\begin{Highlighting}[]
    \NormalTok{lst = [}\DecValTok{1}\NormalTok{,}\DecValTok{2}\NormalTok{] + [}\DecValTok{3}\NormalTok{,}\DecValTok{4}\NormalTok{]}
\end{Highlighting}
\end{Shaded}

In `Haskell this would be very similar:

\begin{Shaded}
\begin{Highlighting}[]
    \NormalTok{lst }\FunctionTok{=} \NormalTok{[}\DecValTok{1}\NormalTok{,}\DecValTok{2}\NormalTok{] }\FunctionTok{++} \NormalTok{[}\DecValTok{3}\NormalTok{,}\DecValTok{4}\NormalTok{]}
\end{Highlighting}
\end{Shaded}

\subsection{Anonymous functions}\label{anonymous-functions}

In \texttt{JavaScript} you can define \emph{anonymous} functions
(functions without a name) such as:

\begin{Shaded}
\begin{Highlighting}[]
    \KeywordTok{var} \NormalTok{f = }\KeywordTok{function}\NormalTok{(x,y)\{}\KeywordTok{return} \NormalTok{x*y+x+y\};}
\end{Highlighting}
\end{Shaded}

In \texttt{Haskell}, such anonymous functions are called \emph{lambda}
functions and they are actually the basis of the language. Again, the
syntax is very compact:

\begin{Shaded}
\begin{Highlighting}[]
    \NormalTok{f }\FunctionTok{=} \NormalTok{\textbackslash{}x y }\OtherTok{->} \NormalTok{x}\FunctionTok{*}\NormalTok{y}\FunctionTok{+}\NormalTok{x}\FunctionTok{+}\NormalTok{y}
\end{Highlighting}
\end{Shaded}

\subsection{Higher-order functions}\label{higher-order-functions}

Finally, in many languages, functions can operate on functions. For
example, in \texttt{Perl} you can modify the elements in a list using:

\begin{Shaded}
\begin{Highlighting}[]
    \FunctionTok{map} \KeywordTok{sub }\NormalTok{(}\DataTypeTok{$}\NormalTok{x)\{}\DataTypeTok{$x}\KeywordTok{*}\DecValTok{2+1}\NormalTok{\}, [}\DecValTok{1}\NormalTok{..}\DecValTok{10}\NormalTok{]}
\end{Highlighting}
\end{Shaded}

\texttt{Haskell} provides many of these so-called \emph{higher-order
functions}, and lets you define your own.

\begin{Shaded}
\begin{Highlighting}[]
    \NormalTok{map (\textbackslash{}x }\OtherTok{->} \NormalTok{x}\FunctionTok{*}\DecValTok{2}\FunctionTok{+}\DecValTok{1}\NormalTok{) [}\DecValTok{1}\FunctionTok{..}\DecValTok{10}\NormalTok{]}
\end{Highlighting}
\end{Shaded}

\end{document}  